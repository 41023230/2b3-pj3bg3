\renewcommand{\baselinestretch}{1.5} %設定行距
\pagenumbering{roman} %設定頁數為羅馬數字
\clearpage  %設定頁數開始編譯
\sectionef
\addcontentsline{toc}{chapter}{摘~~~要} %將摘要加入目錄
\begin{center}
\LARGE\textbf{摘~~要}\\
\end{center}
\begin{flushleft}
\fontsize{14pt}{20pt}\sectionef\hspace{12pt}\quad 本專案旨在進行協同設計,以改良並優化雙輪車或多輪車,並應用於機器人足球比賽中。該專案分為三個階段,其中專案三是對專案二的延續。團隊組成包含8名成員,並使用CAD軟體進行場景和多輪車零組件的設計。透過採用ZmqRemoteAPI Python編程,開發控制程式以支持操控多輪車在足球場景中進行比賽。\\[12pt]

\fontsize{14pt}{20pt}\sectionef\hspace{12pt}\quad 專案三的目標是改進雙輪車或多輪車的行進和對戰效能。為了達到這一目標,團隊需要優化運動控制,使其能夠靈活且精確地操控。同時,引入協同運動策略,使多輪車能夠協調工作,並在比賽中進行合作攻守。為了提高對環境和球場的感知能力,團隊需要適當地選擇和應用感應器和感知系統。\\[12pt]

\fontsize{14pt}{20pt}\sectionef\hspace{12pt}\quad 在控制系統優化的過程中,團隊將進行模擬和測試,並根據實際結果和需求不斷進行改進和優化。最終,團隊將提供相關檔案的下載連結,並製作線上簡報和分組報告,以展示協同設計流程和成果。\\[12pt]

\fontsize{14pt}{20pt}\sectionef\hspace{12pt}\quad 透過這個協同設計的機器人踢足球專案,團隊將獲得實踐協同工作和創新的機會,同時提升雙輪車或多輪車在足球比賽中的性能和效能。這將為未來的機器人技術和運動應用帶來新的發展和應用前景。\\[12pt]
\end{flushleft}
\newpage
%=--------------------Abstract----------------------=%