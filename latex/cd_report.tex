\documentclass[14pt,a4paper]{report}  %紙張設定
\usepackage{xeCJK}%中文字體模組
%\setCJKmainfont{標楷體} %設定中文字體
\setCJKmainfont{MoeStandardKai.ttf}
%\newfontfamily\sectionef{Times New Roman}%設定英文字體
\newfontfamily\sectionef{Nimbus Roman}
\usepackage{enumerate}
\usepackage{amsmath,amssymb}%數學公式、符號
\usepackage{amsfonts} %數學簍空的英文字
\usepackage{graphicx, subfigure}%圖形
\usepackage{fontawesome5} %引用icon
\usepackage{type1cm} %調整字體絕對大小
\usepackage{textpos} %設定文字絕對位置
\usepackage[top=2.5truecm,bottom=2.5truecm,
left=3truecm,right=2.5truecm]{geometry}
\usepackage{titlesec} %目錄標題設定模組
\usepackage{titletoc} %目錄內容設定模組
\usepackage{textcomp} %表格設定模組
\usepackage{multirow} %合併行
%\usepackage{multicol} %合併欄
\usepackage{CJK} %中文模組
\usepackage{CJKnumb} %中文數字模組
\usepackage{wallpaper} %浮水印
\usepackage{listings} %引用程式碼
\usepackage{hyperref} %引用url連結
\usepackage{setspace}
\usepackage{lscape}%設定橫式
\lstset{language=Python, %設定語言
		basicstyle=\fontsize{10pt}{2pt}\selectfont, %設定程式內文字體大小
		frame=lines,	%設定程式框架為線
}
%\usepackage{subcaption}%副圖標
\graphicspath{{./../images/}} %圖片預設讀取路徑
\usepackage{indentfirst} %設定開頭縮排模組
\renewcommand{\figurename}{\Large 圖.} %更改圖片標題名稱
\renewcommand{\tablename}{\Large 表.}
\renewcommand{\lstlistingname}{\Large 程式.} %設定程式標示名稱
\hoffset=-5mm %調整左右邊界
\voffset=-8mm %調整上下邊界
\setlength{\parindent}{3em}%設定首行行距縮排
\usepackage{appendix} %附錄
\usepackage{diagbox}%引用表格
\usepackage{multirow}%表格置中
%\usepackage{number line}
%=------------------更改標題內容----------------------=%
\titleformat{\chapter}[hang]{\center\sectionef\fontsize{20pt}{1pt}\bfseries}{\LARGE 第\CJKnumber{\thechapter}章}{1em}{}[]
\titleformat{\section}[hang]{\sectionef\fontsize{18pt}{2.5pt}\bfseries}{{\thesection}}{0.5em}{}[]
\titleformat{\subsection}[hang]{\sectionef\fontsize{18pt}{2.5pt}\bfseries}{{\thesubsection}}{1em}{}[]
%=------------------更改目錄內容-----------------------=%
\titlecontents{chapter}[11mm]{}{\sectionef\fontsize{18pt}{2.5pt}\bfseries\makebox[3.5em][l]
{第\CJKnumber{\thecontentslabel}章}}{}{\titlerule*[0.7pc]{.}\contentspage}
\titlecontents{section}[18mm]{}{\sectionef\LARGE\makebox[1.5em][l]
{\thecontentslabel}}{}{\titlerule*[0.7pc]{.}\contentspage}
\titlecontents{subsection}[4em]{}{\sectionef\Large\makebox[2.5em][l]{{\thecontentslabel}}}{}{\titlerule*[0.7pc]{.}\contentspage}
%=----------------------章節的間距----------------------=%
\titlespacing*{\chapter} {0pt}{0pt}{18pt}
\titlespacing*{\section} {0pt}{12pt}{6pt}
\titlespacing*{\subsection} {0pt}{6pt}{6pt}
%=----------------------標題-------------------------=%             
\begin{document} %文件
\sectionef %設定英文字體啟用
\vspace{12em}
\begin{titlepage}%開頭
\begin{center}   %標題  
\makebox[1.5\width][s] %[s] 代表 Stretch the interword space in text across the entire width
{\fontsize{24pt}{2.5pt}國立虎尾科技大學}\\[18pt]
\makebox[1.5\width][s]
{\fontsize{24pt}{2.5pt}機械設計工程系}\\[18pt]
\sectionef\fontsize{24pt}{1em}\selectfont\textbf
{
\vspace{0.5em}
cd2023 2b-pj3bg3分組報告}\\[18pt]
%設定文字盒子 [方框寬度的1.5倍寬][對其方式為文字平均分分布於方框中]\\距離下方18pt
\vspace{1em} %下移
\fontsize{30pt}{1pt}\selectfont\textbf{網際足球場景設計}\\
\vspace{1em}
\sectionef\fontsize{30pt}{1em}\selectfont\textbf
{
\vspace{0.5em}
Web-based Football Scene Design}
 \vspace{2em}
%=---------------------參與人員-----------------------=%             
\end{center}
\begin{flushleft}
\begin{LARGE}

\hspace{32mm}\makebox[5cm][s]
{指導教授:\quad 嚴\quad 家\quad 銘\quad 老\quad 師}\\[6pt]
\hspace{32mm}\makebox[5cm][s]
{班\qquad 級:\quad 四\quad 設\quad 二\quad 乙}\\[6pt]
\hspace{32mm}\makebox[5cm][s]
{學\qquad 生:\quad 陳\quad 冠\quad 珉\quad(41023220)}
\\[6pt]
\hspace{32mm}\makebox[5cm][s]
{\hspace{36.5mm}陳\quad 建\quad 霖\quad(41023226)}\\[6pt]
\hspace{32mm}\makebox[5cm][s]
{\hspace{36.5mm}彭\quad 聖\quad 宗\quad(41023230)}\\[6pt]
\hspace{32mm}\makebox[5cm][s]
{\hspace{36.5mm}湛\quad 有\quad 杰\quad(41023231)}\\[6pt]
\hspace{32mm}\makebox[5cm][s]
{\hspace{36.5mm}雲\quad 敬\quad 家\quad(41023232)}\\[6pt]
\hspace{32mm}\makebox[5cm][s]
{\hspace{36.5mm}黃\quad 文\quad 彥\quad(41023233)}\\[6pt]
\hspace{32mm}\makebox[5cm][s]
{\hspace{36.5mm}蔡\quad 叡\quad 得\quad(41023250)}\\[6pt]
\hspace{32mm}\makebox[5cm][s]
{\hspace{36.5mm}謝\quad 宗\quad 銘\quad(41023253)}\\[6pt]
\hspace{32mm}\makebox[5cm][s]

%設定文字盒子[寬度為5cm][對其方式為文字平均分分布於方框中]空白距離{36.5mm}\空白1em
\end{LARGE}
\end{flushleft}
\vspace{6em}
\fontsize{18pt}{2pt}\selectfont\centerline{\makebox[\width][s]
{中華民國\hspace{3em} 
112 \quad 年\quad 6\quad 月}}
\end{titlepage}
\newpage

%=------------------------摘要-----------------------=%
\input{abstract.tex}
\newpage
\renewcommand{\baselinestretch}{1.5} %設定行距

%=------------------------誌謝----------------------=%
\begin{center}	
\addcontentsline{toc}{chapter}{誌~~~謝}
\LARGE\textbf{誌~~謝}\\
\end{center}
\begin{center}
\begin{flushleft}
\fontsize{14pt}{20pt}\sectionef\hspace{12pt}\quad 在此鄭重感謝製作以及協助本分組報告完成的所有人員,首先向嚴家銘老師致謝,他解決我們的各種提問,甚至從來沒有不耐煩,總是貼心為我們找出最佳解答,接著是由OpenAI創造的ChatGPT,在我們需要創作的時候,他總能提供我們源源不斷的創意,最後是由本分組成員同心協力才得以完成本報告,特此感謝

\end{flushleft}
\newpage
%=------------------------目錄----------------------=%
\renewcommand{\contentsname}{\centerline{\fontsize{18pt}{\baselineskip}\selectfont\textbf{目\quad 錄}}}
\tableofcontents  %目錄產生
\newpage

\end{center}
%=-------------------------內容----------------------=%

\chapter{division}

工作內容:\\

球場設計:41023253\\

四輪車設計:41023230 41023231 41023233 41023250 41023253\\

四輪車程式設計:41023226\\

記分板設計:41023232\\

記分板程式設計:41023226\\

得分球員版設計:41023226\\

得分球員版程式設計:41023226\\

pdf ppt製作:41023226\\

整理網頁:41023220\\
\input{2_update.tex}
\input{3_pj3.tex}
\chapter{code}
\section{code\_car}
第一版程式,使用兩邊的速度差來控制方向,且可以在前後的同時控制方向\\
\begin{lstlisting}[language=Python, frame=single, numbers=left, captionpos=b, basicstyle=\ttfamily\small, showstringspaces=false, breaklines=true, tabsize=4, xleftmargin=15pt]
from zmqRemoteApi_IPv6 import RemoteAPIClient
import keyboard
 
client = RemoteAPIClient('localhost', 23000)
 
print('Program started')
sim = client.getObject('sim')
sim.startSimulation()
print('Simulation started')
 
def setWheelMotion(leftSpeed, rightSpeed):
    # Set target velocity for each wheel
    frontLeftWheel = sim.getObject('/frontLeftJoint')
    frontRightWheel = sim.getObject('/frontRightJoint')
    rearLeftWheel = sim.getObject('/rearLeftJoint')
    rearRightWheel = sim.getObject('/rearRightJoint')
    sim.setJointTargetVelocity(frontLeftWheel, leftSpeed)
    sim.setJointTargetVelocity(frontRightWheel, rightSpeed)
    sim.setJointTargetVelocity(rearLeftWheel, leftSpeed)
    sim.setJointTargetVelocity(rearRightWheel, rightSpeed)
 
# Initialize motion variables
leftSpeed = 0
rightSpeed = 0
 
# Main loop
while True:
    # Check keyboard input
    if keyboard.is_pressed('w'):
        leftSpeed = -10  # Forward motion
        rightSpeed = -10  # Forward motion
    elif keyboard.is_pressed('s'):
        leftSpeed = 10  # Backward motion
        rightSpeed = 10  # Backward motion
    else:
        leftSpeed = 0
        rightSpeed = 0
 
    if keyboard.is_pressed('a'):
        leftSpeed += 5  # Left turn
        rightSpeed -= 5  # Left turn
    elif keyboard.is_pressed('d'):
        leftSpeed -= 5  # Right turn
        rightSpeed += 5  # Right turn
         
    if keyboard.is_pressed('q'):
        break  # Quit
 
    # Set motion for all wheels
    setWheelMotion(leftSpeed, rightSpeed)
 
# Stop the simulation
sim.stopSimulation()
\end{lstlisting}
第二版程式,前輪各多加一個joint,使其運動更加合理\\
\begin{lstlisting}[language=Python, frame=single, numbers=left, captionpos=b, basicstyle=\ttfamily\small, showstringspaces=false, breaklines=true, tabsize=4, xleftmargin=15pt]
from zmqRemoteApi_IPv6 import RemoteAPIClient
import keyboard
 
client = RemoteAPIClient('localhost', 23000)
 
sim = client.getObject('sim')
 
sim.startSimulation()
 
frontLeftSteeringJoint = sim.getObject('/frontLeftJoint1')
frontRightSteeringJoint = sim.getObject('/frontRightJoint1')
frontLeftWheel = sim.getObject('/frontLeftJoint2')
frontRightWheel = sim.getObject('/frontRightJoint2')
rearLeftWheel = sim.getObject('/rearLeftJoint')
rearRightWheel = sim.getObject('/rearRightJoint')
 
def setFrontWheelSteeringAngle(steeringAngle):
    sim.setJointTargetPosition(frontLeftSteeringJoint, steeringAngle)
    sim.setJointTargetPosition(frontRightSteeringJoint, steeringAngle)
 
def setAllWheelSpeed(speed):
    sim.setJointTargetVelocity(frontLeftWheel, speed)
    sim.setJointTargetVelocity(frontRightWheel, speed)
    sim.setJointTargetVelocity(rearLeftWheel, speed)
    sim.setJointTargetVelocity(rearRightWheel, speed)
 
steeringAngle = 0
speed = 0
 
SPEED_FORWARD = -20
SPEED_BACKWARD = 20
STEERING_ANGLE_LEFT = 0.3
STEERING_ANGLE_RIGHT = -0.3
 
while True:
    if keyboard.is_pressed('w'):
        speed = SPEED_FORWARD
    elif keyboard.is_pressed('s'):
        speed = SPEED_BACKWARD
    else:
        speed = 0
 
    if keyboard.is_pressed('a'):
        steeringAngle = STEERING_ANGLE_LEFT
    elif keyboard.is_pressed('d'):
        steeringAngle = STEERING_ANGLE_RIGHT
    else:
        steeringAngle = 0
 
    if keyboard.is_pressed('q'):
        break 
 
    setFrontWheelSteeringAngle(steeringAngle)
     
    setAllWheelSpeed(speed)
 
sim.stopSimulation()
\end{lstlisting}
\section{code\_scoreboard}
\begin{lstlisting}[language=Python, frame=single, numbers=left, captionpos=b, basicstyle=\ttfamily\small, showstringspaces=false, breaklines=true, tabsize=4, xleftmargin=15pt]
function sysCall_init()
    score = 0
    wheelJoint = sim.getObjectHandle('/joint1g')
    robot = {
        sim.getObjectHandle('/car1'),
        sim.getObjectHandle('/car2'),
        sim.getObjectHandle('/car3'),
        sim.getObjectHandle('/car4'),
        sim.getObjectHandle('/car5'),
        sim.getObjectHandle('/car6'),
        sim.getObjectHandle('/car7'),
        sim.getObjectHandle('/car8')
    }
    initialPos = {
        {-1.050, -0.77134, 0.21},
        {-1.050, -0.27134, 0.21},
        {-1.050, 0.22867, 0.21},
        {-1.050, 0.62866, 0.21},
        {1.175, -0.77134, 0.21},
        {1.175, -0.27134, 0.21},
        {1.175, 0.22867, 0.21},
        {1.175, 0.62866, 0.21}
    }
    initialOri = {
        {0, 90, 0},
        {0, 90, 0},
        {0, 90, 0},
        {0, 90, 0},
        {0, -90, 180},
        {0, -90, 180},
        {0, -90, 180},
        {0, -90, 180}
    }
end
 
sensor = sim.getObject('./sensor')
initialPosBall = sim.getObjectPosition(sensor, -1)
ball = sim.getObject('/ball')
 
function sysCall_actuation()
    result = sim.readProximitySensor(sensor)
    if (result > 0) then
        score = score + 1
        sim.setObjectPosition(ball, -1, {0,0,0.25})
         
        -- Rotate the wheel joint by 36 degrees
        local currentAngle = sim.getJointPosition(wheelJoint)
        local targetAngle = currentAngle + math.rad(-36)
        sim.setJointTargetPosition(wheelJoint, targetAngle)
         
        for i = 1, 8 do
            sim.setObjectPosition(robot[i], -1, initialPos[i])
            sim.setObjectOrientation(robot[i], -1, initialOri[i])
        end
    end
end
\end{lstlisting}
這個程式控制了球進入球門後,記分板轉動36度,以及車子與球回到初始位置,目前發生未知原因使的5到8的車子在球進入球門後,位置與方向與我所設置的不相符\\
\section{code\_scorer version}
將進球後顯示goal加入記分板程式中\\
\begin{lstlisting}[language=Python, frame=single, numbers=left, captionpos=b, basicstyle=\ttfamily\small, showstringspaces=false, breaklines=true, tabsize=4, xleftmargin=15pt]
function sysCall_init()
    score = 0
    wheelJoint = sim.getObjectHandle('/joint1g')
    robot = {
        sim.getObjectHandle('/car1'),
        sim.getObjectHandle('/car2'),
        sim.getObjectHandle('/car3'),
        sim.getObjectHandle('/car4'),
        sim.getObjectHandle('/car5'),
        sim.getObjectHandle('/car6'),
        sim.getObjectHandle('/car7'),
        sim.getObjectHandle('/car8')
    }
    initialPos = {
        {-1.050, -0.77134, 0.21},
        {-1.050, -0.27134, 0.21},
        {-1.050, 0.22867, 0.21},
        {-1.050, 0.62866, 0.21},
        {1.175, -0.77134, 0.21},
        {1.175, -0.27134, 0.21},
        {1.175, 0.22867, 0.21},
        {1.175, 0.62866, 0.21}
    }
    initialOri = {
        {0, 90, 0},
        {0, 90, 0},
        {0, 90, 0},
        {0, 90, 0},
        {0, -90, 180},
        {0, -90, 180},
        {0, -90, 180},
        {0, -90, 180}
    }
    displayHandles = {
        sim.getObjectHandle('/display1'),
        sim.getObjectHandle('/display2'),
        sim.getObjectHandle('/display3'),
        sim.getObjectHandle('/display4'),
        sim.getObjectHandle('/display5'),
        sim.getObjectHandle('/display6'),
        sim.getObjectHandle('/display7')
    }
    displayParameters = {
        {0, 0, 1, 0, 0, 1, 0},
        {1, 0, 1, 1, 1, 0, 1},
        {1, 0, 1, 1, 0, 1, 1},
        {0, 1, 1, 1, 0, 1, 0},
        {1, 1, 0, 1, 0, 1, 1},
        {1, 1, 0, 1, 1, 1, 1},
        {1, 0, 1, 0, 0, 1, 0},
        {1, 1, 1, 1, 1, 1, 1}
    }
    goalEffectEnabled = false
    lastGoalPlayer = nil
    flashingState = false
    flashingTimer = 0
    goalHandle = sim.getObjectHandle('/goal')
    flashingDuration = 0.5
    flashingTimer = 0
    isFlashing = false
end
 
sensor = sim.getObject('./sensor')
initialPosBall = sim.getObjectPosition(sensor, -1)
ball = sim.getObject('/ball')
 
function sysCall_actuation()
    result = sim.readProximitySensor(sensor)
    if (result > 0) then
        score = score + 1
        sim.setObjectPosition(ball, -1, {0,0,0.25})
         
        -- Rotate the wheel joint by 36 degrees
        local currentAngle = sim.getJointPosition(wheelJoint)
        local targetAngle = currentAngle + math.rad(-36)
        sim.setJointTargetPosition(wheelJoint, targetAngle)
         
        for i = 1, 8 do
            sim.setObjectPosition(robot[i], -1, initialPos[i])
            sim.setObjectOrientation(robot[i], -1, initialOri[i])
        end
         
        goalEffectEnabled = true
        lastGoalPlayer = result
        flashingState = false
        flashingTimer = 0
    end
     
    if goalEffectEnabled then
        if not isFlashing then
            flashingTimer = flashingTimer + sim.getSimulationTimeStep()
            if flashingTimer >= flashingDuration then
                isFlashing = true
                sim.setShapeColor(goalHandle, nil, sim.colorcomponent_ambient_diffuse, {1, 0.9, 0})
            end
        elseif isFlashing then
            flashingTimer = flashingTimer + sim.getSimulationTimeStep()
            if flashingTimer >= flashingDuration + 3 then
                goalEffectEnabled = false
                isFlashing = false
                sim.setShapeColor(goalHandle, nil, sim.colorcomponent_ambient_diffuse, {0, 0, 0})
            end
        end
    end
end
 
function sysCall_cleanup()
    sim.setShapeColor(goalHandle, nil, sim.colorcomponent_ambient_diffuse,{0, 0, 0})
end
\end{lstlisting}
檢測與球碰撞的球員並顯示其背號在七段顯示器中\\
\begin{lstlisting}[language=Python, frame=single, numbers=left, captionpos=b, basicstyle=\ttfamily\small, showstringspaces=false, breaklines=true, tabsize=4, xleftmargin=15pt]
local robotHandles = {}
local ballHandle = -1
local displayHandles = {}
local displayParameters = {
    {0, 0, 1, 0, 0, 1, 0},
    {1, 0, 1, 1, 1, 0, 1},
    {1, 0, 1, 1, 0, 1, 1},
    {0, 1, 1, 1, 0, 1, 0},
    {1, 1, 0, 1, 0, 1, 1},
    {1, 1, 0, 1, 1, 1, 1},
    {1, 0, 1, 0, 0, 1, 0},
    {1, 1, 1, 1, 1, 1, 1}
}
 
function sysCall_init()
    for i = 1, 8 do
        robotHandles[i] = sim.getObjectHandle('car' .. i)
    end
    ballHandle = sim.getObjectHandle('ball')
 
    for i = 1, 7 do
        displayHandles[i] = sim.getObjectHandle('/display' .. i)
    end
end
 
function sysCall_actuation()
    for i = 1, 8 do
        local collision = sim.checkCollision(robotHandles[i], ballHandle)
        if collision ~= 0 then
            setDisplayDigit(i, i <= 4)
            break
        end
    end
end
 
function setDisplayDigit(digit, isRed)
    for i = 1, 7 do
        if displayParameters[digit][i] == 1 then
            if isRed then
                sim.setShapeColor(displayHandles[i], nil, sim.colorcomponent_ambient_diffuse, {1, 0, 0})
            else
                sim.setShapeColor(displayHandles[i], nil, sim.colorcomponent_ambient_diffuse, {0, 1, 0})
            end
        else
            sim.setShapeColor(displayHandles[i], nil, sim.colorcomponent_ambient_diffuse, {0, 0, 0})
        end
    end
end
\end{lstlisting}
\chapter{image}



\section{image\_Quadricycle First Edition}
\section{Quadricycle First Edition\_41023230機器人}
\begin{figure}
\includegraphics[width=3.75in]{41023230robot1}
\end{figure}
\section{Quadricycle First Edition\_41023231機器人}
\begin{figure}
\includegraphics[width=3.75in]{41023231robot1}
\end{figure}
\section{Quadricycle First Edition\_41023232機器人}
\begin{figure}
\includegraphics[width=3.75in]{41023232robot1}
\end{figure}
\section{Quadricycle First Edition\_41023233機器人}
\begin{figure}
\includegraphics[width=3.75in]{41023233robot1}
\end{figure}
\section{Quadricycle First Edition\_41023253機器人}
\begin{figure}
\includegraphics[width=3.75in]{41023253robot1}
\end{figure}
\section{image\_Quadricycle Second Edition}
\section{Quadricycle Second Edition\_41023233機器人}
\begin{figure}
\includegraphics[width=3.75in]{41023233robot2}
\end{figure}
\section{Quadricycle Second Edition\_41023250機器人}
\begin{figure}
\includegraphics[width=3.75in]{41023250robot2}
\end{figure}
\section{image\_court}
\section{court\_球場本體}
\begin{figure}
\includegraphics[width=3.75in]{court}
\end{figure}
\section{court\_球門}
\begin{figure}
\includegraphics[width=3.75in]{goal}
\end{figure}
\section{court\_球場球門組合}
\begin{figure}
\includegraphics[width=3.75in]{court+goal}
\end{figure}
\section{court\_組合並加入車子}
\begin{figure}
\includegraphics[width=3.75in]{race}
\end{figure}
\section{court\_Independence Stadium}
\begin{figure}
\includegraphics[width=3.75in]{Independence Stadium}
\end{figure}
\section{image\_scoreboard}
\begin{figure}
\includegraphics[width=3.75in]{scoreboard}
\end{figure}
\section{image\_race}
\begin{figure}
\includegraphics[width=5in]{race2}
\end{figure}




%=---------------------參考文獻----------------------=%

%=---------------附錄-----------------=%
\addcontentsline{toc}{chapter}{附錄} %新增目錄名稱

\newpage
%=-------------作者簡介-----------------=%
    \addcontentsline{toc}{chapter}{作者簡介}
    \begin{center}
	\fontsize{20pt}{0em}\selectfont \bf{作者簡介}\\
	\end{center}	
	{\begin{textblock}{6}(0,0.5)
	\begin{figure}
	\includegraphics[width=1.25in]{41023220}  %作者照片
	\end{figure}
	\end{textblock}}
	{\renewcommand\baselinestretch{0.99}\selectfont %設定以下行距
	{\begin{textblock}{15}(3.5,0.7)%{寬度}(以左上角為原點之右移量,下移量)
	\noindent\fontsize{14pt}{0em}\selectfont \makebox[4em][s]{姓名}\enspace:\enspace
    \fontsize{14pt}{0em}\selectfont \makebox[4em][s]{陳冠珉}\\     \hspace*{\fill} \\
    \fontsize{14pt}{0em}\selectfont \makebox[4em][s]{學號}\enspace:\enspace
    \fontsize{14pt}{0em}\selectfont \makebox[4em][s]{41023220} \\ %\makebox為文本盒子
    \hspace*{\fill} \\
    \fontsize{14pt}{0em}\selectfont \makebox[4em][s]{就讀學校}\enspace:\enspace
    \fontsize{14pt}{0em}\selectfont \makebox[9em][s]{國立虎尾科技大學}\\
    \fontsize{14pt}{0em}\selectfont \makebox[5em][s]{\quad}\enspace\enspace
    \fontsize{14pt}{0em}\selectfont \makebox[8em][s]{機械設計工程系}\\
    \hspace*{\fill} \\
    \fontsize{14pt}{0em}\selectfont \makebox[4em][s]{經歷}\enspace:\enspace
    \end{textblock}}}
   % \hspace*{\fill} \\
   \vspace{2em}
	{\begin{textblock}{6}(0,2.3)
	\begin{figure}
	\includegraphics[width=1.15in]{41023226}  %作者照片
    \end{figure}
    \end{textblock}}
    {\renewcommand\baselinestretch{0.99}
    \selectfont %設定以下行距
    {\begin{textblock}{15}(3.5,2.5) %{寬度}(以左上角為原點之右移量,下移量)
\noindent\fontsize{14pt}{0em}\selectfont \makebox[4em][s]{姓名}\enspace:\enspace
\fontsize{14pt}{0em}\selectfont \makebox[4em][s]{陳建霖}\\ 
\hspace*{\fill} \\
\fontsize{14pt}{0em}\selectfont \makebox[4em][s]{學號}\enspace:\enspace
\noindent\fontsize{14pt}{0em}\selectfont \makebox[4em][s]{41023226} \\ 
\hspace*{\fill} \\
\fontsize{14pt}{0em}\selectfont \makebox[4em][s]{就讀學校}\enspace:\enspace
\fontsize{14pt}{0em}\selectfont \makebox[9em][s]{國立虎尾科技大學}\\
\fontsize{14pt}{0em}\selectfont \makebox[5em][s]{\quad}\enspace\enspace
\fontsize{14pt}{0em}\selectfont \makebox[8em][s]{機械設計工程系}\\
\hspace*{\fill} \\
\fontsize{14pt}{0em}\selectfont \makebox[4em][s]{經歷}\enspace:\enspace
    \end{textblock}}}
   % \hspace*{\fill} \\
   \vspace{2em}
	{\begin{textblock}{6}(0,4.1)
	\begin{figure}
	\includegraphics[width=1.15in]{41023230}  %作者照片
    \end{figure}
    \end{textblock}}
    {\renewcommand\baselinestretch{0.99}
    \selectfont %設定以下行距
    {\begin{textblock}{15}(3.5,4.3) %{寬度}(以左上角為原點之右移量,下移量)
\noindent\fontsize{14pt}{0em}\selectfont \makebox[4em][s]{姓名}\enspace:\enspace
\fontsize{14pt}{0em}\selectfont \makebox[4em][s]{彭聖宗}\\ 
\hspace*{\fill} \\
\fontsize{14pt}{0em}\selectfont \makebox[4em][s]{學號}\enspace:\enspace
\noindent\fontsize{14pt}{0em}\selectfont \makebox[4em][s]{41023230} \\ 
\hspace*{\fill} \\
\fontsize{14pt}{0em}\selectfont \makebox[4em][s]{就讀學校}\enspace:\enspace
\fontsize{14pt}{0em}\selectfont \makebox[9em][s]{國立虎尾科技大學}\\
\fontsize{14pt}{0em}\selectfont \makebox[5em][s]{\quad}\enspace\enspace
\fontsize{14pt}{0em}\selectfont \makebox[8em][s]{機械設計工程系}\\
\hspace*{\fill} \\
\fontsize{14pt}{0em}\selectfont \makebox[4em][s]{經歷}\enspace:\enspace
    \end{textblock}}}
   % \hspace*{\fill} \\
   \vspace{2em}
	{\begin{textblock}{6}(0,5.9)
	\begin{figure}
	\includegraphics[width=1.15in]{41023231}  %作者照片
    \end{figure}
    \end{textblock}}
    {\renewcommand\baselinestretch{0.99}
    \selectfont %設定以下行距
    {\begin{textblock}{15}(3.5,6.1) %{寬度}(以左上角為原點之右移量,下移量)
\noindent\fontsize{14pt}{0em}\selectfont \makebox[4em][s]{姓名}\enspace:\enspace
\fontsize{14pt}{0em}\selectfont \makebox[4em][s]{湛有杰}\\ 
\hspace*{\fill} \\
\fontsize{14pt}{0em}\selectfont \makebox[4em][s]{學號}\enspace:\enspace
\noindent\fontsize{14pt}{0em}\selectfont \makebox[4em][s]{41023231} \\ 
\hspace*{\fill} \\
\fontsize{14pt}{0em}\selectfont \makebox[4em][s]{就讀學校}\enspace:\enspace
\fontsize{14pt}{0em}\selectfont \makebox[9em][s]{國立虎尾科技大學}\\
\fontsize{14pt}{0em}\selectfont \makebox[5em][s]{\quad}\enspace\enspace
\fontsize{14pt}{0em}\selectfont \makebox[8em][s]{機械設計工程系}\\
\hspace*{\fill} \\
\fontsize{14pt}{0em}\selectfont \makebox[4em][s]{經歷}\enspace:\enspace
    \end{textblock}}}
   % \hspace*{\fill} \\
   \vspace{2em}
	{\begin{textblock}{6}(0,7.7)
	\begin{figure}
	\includegraphics[width=1.15in]{41023232}  %作者照片
    \end{figure}
    \end{textblock}}
    {\renewcommand\baselinestretch{0.99}
    \selectfont %設定以下行距
    {\begin{textblock}{15}(3.5,7.9) %{寬度}(以左上角為原點之右移量,下移量)
\noindent\fontsize{14pt}{0em}\selectfont \makebox[4em][s]{姓名}\enspace:\enspace
\fontsize{14pt}{0em}\selectfont \makebox[4em][s]{雲敬家}\\ 
\hspace*{\fill} \\
\fontsize{14pt}{0em}\selectfont \makebox[4em][s]{學號}\enspace:\enspace
\noindent\fontsize{14pt}{0em}\selectfont \makebox[4em][s]{41023233} \\ 
\hspace*{\fill} \\
\fontsize{14pt}{0em}\selectfont \makebox[4em][s]{就讀學校}\enspace:\enspace
\fontsize{14pt}{0em}\selectfont \makebox[9em][s]{國立虎尾科技大學}\\
\fontsize{14pt}{0em}\selectfont \makebox[5em][s]{\quad}\enspace\enspace
\fontsize{14pt}{0em}\selectfont \makebox[8em][s]{機械設計工程系}\\
\hspace*{\fill} \\
\fontsize{14pt}{0em}\selectfont \makebox[4em][s]{經歷}\enspace:\enspace
    \end{textblock}}}
    %\hspace*{\fill} \\

\newpage
%=-------------作者簡介2-----------------=%
    \addcontentsline{toc}{chapter}{作者簡介2}
    \begin{center}
	\fontsize{20pt}{0em}\selectfont \bf{作者簡介2}\\
	\end{center}	
    %\hspace*{\fill} \\
    \vspace{2em}
    {\begin{textblock}{6}(0,0.5)
    \begin{figure}
        \includegraphics[width=1.15in]{41023233} %{}內是圖片文件的相對路徑
    \end{figure}
    \end{textblock}}
    {\renewcommand\baselinestretch{0.99}\selectfont %設定以下行距
    {\begin{textblock}{15}(3.5,0.7) %{寬度}(以左上角為原點之右移量,下移量)
\noindent\fontsize{14pt}{0em}\selectfont \makebox[4em][s]{姓名}\enspace:\enspace%\noindent指定首行不進行縮排
\fontsize{14pt}{0em}\selectfont \makebox[4em][s]{黃文彥}\\ 
\hspace*{\fill} \\
\noindent\fontsize{14pt}{0em}\selectfont \makebox[4em][s]{學號}\enspace:\enspace
\noindent\fontsize{14pt}{0em}\selectfont \makebox[4em][s]{41023233} \\ %\makebox為文本盒子
\hspace*{\fill} \\
\noindent\fontsize{14pt}{0em}\selectfont \makebox[4em][s]{就讀學校}\enspace:\enspace
\noindent\fontsize{14pt}{0em}\selectfont \makebox[9em][s]{國立虎尾科技大學}\\
\noindent\fontsize{14pt}{0em}\selectfont \makebox[5em][s]{\quad}\enspace\enspace
\noindent\fontsize{14pt}{0em}\selectfont \makebox[8em][s]{機械設計工程系}\\
\hspace*{\fill} \\
\noindent\fontsize{14pt}{0em}\selectfont \makebox[4em][s]{經歷}\enspace:\enspace
    \end{textblock}}}
   % \hspace*{\fill} \\
   \vspace{2em}
	{\begin{textblock}{6}(0,2.3)
	\begin{figure}
	\includegraphics[width=1.15in]{41023250}  %作者照片
    \end{figure}
    \end{textblock}}
    {\renewcommand\baselinestretch{0.99}
    \selectfont %設定以下行距
    {\begin{textblock}{15}(3.5,2.5) %{寬度}(以左上角為原點之右移量,下移量)
\noindent\fontsize{14pt}{0em}\selectfont \makebox[4em][s]{姓名}\enspace:\enspace
\fontsize{14pt}{0em}\selectfont \makebox[4em][s]{蔡叡得}\\ 
\hspace*{\fill} \\
\fontsize{14pt}{0em}\selectfont \makebox[4em][s]{學號}\enspace:\enspace
\noindent\fontsize{14pt}{0em}\selectfont \makebox[4em][s]{41023250} \\ 
\hspace*{\fill} \\
\fontsize{14pt}{0em}\selectfont \makebox[4em][s]{就讀學校}\enspace:\enspace
\fontsize{14pt}{0em}\selectfont \makebox[9em][s]{國立虎尾科技大學}\\
\fontsize{14pt}{0em}\selectfont \makebox[5em][s]{\quad}\enspace\enspace
\fontsize{14pt}{0em}\selectfont \makebox[8em][s]{機械設計工程系}\\
\hspace*{\fill} \\
\fontsize{14pt}{0em}\selectfont \makebox[4em][s]{經歷}\enspace:\enspace
    \end{textblock}}}
   % \hspace*{\fill} \\
   \vspace{2em}
    {\begin{textblock}{6}(0,4.1)
    \begin{figure}
        \includegraphics[width=1.15in]{41023253} %{}內是圖片文件的相對路徑
    \end{figure}
    \end{textblock}}
    {\renewcommand\baselinestretch{0.99}\selectfont %設定以下行距
    {\begin{textblock}{15}(3.5,4.3) %{寬度}(以左上角為原點之右移量,下移量)
\noindent\noindent\fontsize{14pt}{0em}\selectfont \makebox[4em][s]{姓名}\enspace:\enspace
\noindent\fontsize{14pt}{0em}\selectfont \makebox[4em][s]{謝宗銘}\\ \hspace*{\fill} \\
\noindent\fontsize{14pt}{0em}\selectfont \makebox[4em][s]{學號}\enspace:\enspace
\noindent\fontsize{14pt}{0em}\selectfont \makebox[4em][s]{41023253} \\ \hspace*{\fill} \\
\noindent\fontsize{14pt}{0em}\selectfont \makebox[4em][s]{就讀學校}\enspace:\enspace
\noindent\fontsize{14pt}{0em}\selectfont \makebox[9em][s]{國立虎尾科技大學}\\
\noindent\fontsize{14pt}{0em}\selectfont \makebox[5em][s]{\quad}\enspace\enspace
\noindent\fontsize{14pt}{0em}\selectfont \makebox[8em][s]{機械設計工程系}\\
\hspace*{\fill} \\
\noindent\fontsize{14pt}{0em}\selectfont \makebox[4em][s]{經歷}\enspace:\enspace
    \end{textblock}}}
    %\hspace*{\fill} \\

\newpage
%=----------------書背----------------------=%
\pagestyle{empty}%設定沒有頁眉和頁腳
\begin{center}
\fontsize{0.001pt}{1pt}\selectfont .\\
\vspace{4em}
\fontsize{30pt}{30pt}\selectfont 【13】 \\
\fontsize{20pt}{20pt}\selectfont
\vspace{0.5em}
分\\
類\\
編\\
號\\
\vspace{0.5em}
\hspace{-0.5em}:\\
\vspace{0.5em}
\rotatebox[origin=cc]{270}{\sectionef\LARGE \textbf{pj3bg3}}\\ %旋轉
\vspace{0.5em}
網\\
際\\
足\\
球\\
場\\
景\\
設\\
計\\
\vspace{2em}
一\\
一\\
二\\
級\\

\end{center}
%\newpage
%\begin{landscape}  %橫式環境
%\begin{center}
%\fontsize{0.001pt}{1pt}\selectfont .
%\vspace{70mm}
%\rotatebox[origin=cc]{90}{\LARGE 【14】}\rotatebox[origin=cc]%{180}{\LARGE 1-2-APP-8765} %旋轉
%\end{center}
%\end{landscape}
\end{document}
