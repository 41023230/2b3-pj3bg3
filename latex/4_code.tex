\chapter{code}
\section{code\_car}
第一版程式,使用兩邊的速度差來控制方向,且可以在前後的同時控制方向\\
\begin{lstlisting}[language=Python, frame=single, numbers=left, captionpos=b, basicstyle=\ttfamily\small, showstringspaces=false, breaklines=true, tabsize=4, xleftmargin=15pt]
from zmqRemoteApi_IPv6 import RemoteAPIClient
import keyboard
 
client = RemoteAPIClient('localhost', 23000)
 
print('Program started')
sim = client.getObject('sim')
sim.startSimulation()
print('Simulation started')
 
def setWheelMotion(leftSpeed, rightSpeed):
    # Set target velocity for each wheel
    frontLeftWheel = sim.getObject('/frontLeftJoint')
    frontRightWheel = sim.getObject('/frontRightJoint')
    rearLeftWheel = sim.getObject('/rearLeftJoint')
    rearRightWheel = sim.getObject('/rearRightJoint')
    sim.setJointTargetVelocity(frontLeftWheel, leftSpeed)
    sim.setJointTargetVelocity(frontRightWheel, rightSpeed)
    sim.setJointTargetVelocity(rearLeftWheel, leftSpeed)
    sim.setJointTargetVelocity(rearRightWheel, rightSpeed)
 
# Initialize motion variables
leftSpeed = 0
rightSpeed = 0
 
# Main loop
while True:
    # Check keyboard input
    if keyboard.is_pressed('w'):
        leftSpeed = -10  # Forward motion
        rightSpeed = -10  # Forward motion
    elif keyboard.is_pressed('s'):
        leftSpeed = 10  # Backward motion
        rightSpeed = 10  # Backward motion
    else:
        leftSpeed = 0
        rightSpeed = 0
 
    if keyboard.is_pressed('a'):
        leftSpeed += 5  # Left turn
        rightSpeed -= 5  # Left turn
    elif keyboard.is_pressed('d'):
        leftSpeed -= 5  # Right turn
        rightSpeed += 5  # Right turn
         
    if keyboard.is_pressed('q'):
        break  # Quit
 
    # Set motion for all wheels
    setWheelMotion(leftSpeed, rightSpeed)
 
# Stop the simulation
sim.stopSimulation()
\end{lstlisting}
第二版程式,前輪各多加一個joint,使其運動更加合理\\
\begin{lstlisting}[language=Python, frame=single, numbers=left, captionpos=b, basicstyle=\ttfamily\small, showstringspaces=false, breaklines=true, tabsize=4, xleftmargin=15pt]
from zmqRemoteApi_IPv6 import RemoteAPIClient
import keyboard
 
client = RemoteAPIClient('localhost', 23000)
 
sim = client.getObject('sim')
 
sim.startSimulation()
 
frontLeftSteeringJoint = sim.getObject('/frontLeftJoint1')
frontRightSteeringJoint = sim.getObject('/frontRightJoint1')
frontLeftWheel = sim.getObject('/frontLeftJoint2')
frontRightWheel = sim.getObject('/frontRightJoint2')
rearLeftWheel = sim.getObject('/rearLeftJoint')
rearRightWheel = sim.getObject('/rearRightJoint')
 
def setFrontWheelSteeringAngle(steeringAngle):
    sim.setJointTargetPosition(frontLeftSteeringJoint, steeringAngle)
    sim.setJointTargetPosition(frontRightSteeringJoint, steeringAngle)
 
def setAllWheelSpeed(speed):
    sim.setJointTargetVelocity(frontLeftWheel, speed)
    sim.setJointTargetVelocity(frontRightWheel, speed)
    sim.setJointTargetVelocity(rearLeftWheel, speed)
    sim.setJointTargetVelocity(rearRightWheel, speed)
 
steeringAngle = 0
speed = 0
 
SPEED_FORWARD = -20
SPEED_BACKWARD = 20
STEERING_ANGLE_LEFT = 0.3
STEERING_ANGLE_RIGHT = -0.3
 
while True:
    if keyboard.is_pressed('w'):
        speed = SPEED_FORWARD
    elif keyboard.is_pressed('s'):
        speed = SPEED_BACKWARD
    else:
        speed = 0
 
    if keyboard.is_pressed('a'):
        steeringAngle = STEERING_ANGLE_LEFT
    elif keyboard.is_pressed('d'):
        steeringAngle = STEERING_ANGLE_RIGHT
    else:
        steeringAngle = 0
 
    if keyboard.is_pressed('q'):
        break 
 
    setFrontWheelSteeringAngle(steeringAngle)
     
    setAllWheelSpeed(speed)
 
sim.stopSimulation()
\end{lstlisting}
\section{code\_scoreboard}
\begin{lstlisting}[language=Python, frame=single, numbers=left, captionpos=b, basicstyle=\ttfamily\small, showstringspaces=false, breaklines=true, tabsize=4, xleftmargin=15pt]
function sysCall_init()
    score = 0
    wheelJoint = sim.getObjectHandle('/joint1g')
    robot = {
        sim.getObjectHandle('/car1'),
        sim.getObjectHandle('/car2'),
        sim.getObjectHandle('/car3'),
        sim.getObjectHandle('/car4'),
        sim.getObjectHandle('/car5'),
        sim.getObjectHandle('/car6'),
        sim.getObjectHandle('/car7'),
        sim.getObjectHandle('/car8')
    }
    initialPos = {
        {-1.050, -0.77134, 0.21},
        {-1.050, -0.27134, 0.21},
        {-1.050, 0.22867, 0.21},
        {-1.050, 0.62866, 0.21},
        {1.175, -0.77134, 0.21},
        {1.175, -0.27134, 0.21},
        {1.175, 0.22867, 0.21},
        {1.175, 0.62866, 0.21}
    }
    initialOri = {
        {0, 90, 0},
        {0, 90, 0},
        {0, 90, 0},
        {0, 90, 0},
        {0, -90, 180},
        {0, -90, 180},
        {0, -90, 180},
        {0, -90, 180}
    }
end
 
sensor = sim.getObject('./sensor')
initialPosBall = sim.getObjectPosition(sensor, -1)
ball = sim.getObject('/ball')
 
function sysCall_actuation()
    result = sim.readProximitySensor(sensor)
    if (result > 0) then
        score = score + 1
        sim.setObjectPosition(ball, -1, {0,0,0.25})
         
        -- Rotate the wheel joint by 36 degrees
        local currentAngle = sim.getJointPosition(wheelJoint)
        local targetAngle = currentAngle + math.rad(-36)
        sim.setJointTargetPosition(wheelJoint, targetAngle)
         
        for i = 1, 8 do
            sim.setObjectPosition(robot[i], -1, initialPos[i])
            sim.setObjectOrientation(robot[i], -1, initialOri[i])
        end
    end
end
\end{lstlisting}
這個程式控制了球進入球門後,記分板轉動36度,以及車子與球回到初始位置,目前發生未知原因使的5到8的車子在球進入球門後,位置與方向與我所設置的不相符\\
\section{code\_scorer version}
將進球後顯示goal加入記分板程式中\\
\begin{lstlisting}[language=Python, frame=single, numbers=left, captionpos=b, basicstyle=\ttfamily\small, showstringspaces=false, breaklines=true, tabsize=4, xleftmargin=15pt]
function sysCall_init()
    score = 0
    wheelJoint = sim.getObjectHandle('/joint1g')
    robot = {
        sim.getObjectHandle('/car1'),
        sim.getObjectHandle('/car2'),
        sim.getObjectHandle('/car3'),
        sim.getObjectHandle('/car4'),
        sim.getObjectHandle('/car5'),
        sim.getObjectHandle('/car6'),
        sim.getObjectHandle('/car7'),
        sim.getObjectHandle('/car8')
    }
    initialPos = {
        {-1.050, -0.77134, 0.21},
        {-1.050, -0.27134, 0.21},
        {-1.050, 0.22867, 0.21},
        {-1.050, 0.62866, 0.21},
        {1.175, -0.77134, 0.21},
        {1.175, -0.27134, 0.21},
        {1.175, 0.22867, 0.21},
        {1.175, 0.62866, 0.21}
    }
    initialOri = {
        {0, 90, 0},
        {0, 90, 0},
        {0, 90, 0},
        {0, 90, 0},
        {0, -90, 180},
        {0, -90, 180},
        {0, -90, 180},
        {0, -90, 180}
    }
    displayHandles = {
        sim.getObjectHandle('/display1'),
        sim.getObjectHandle('/display2'),
        sim.getObjectHandle('/display3'),
        sim.getObjectHandle('/display4'),
        sim.getObjectHandle('/display5'),
        sim.getObjectHandle('/display6'),
        sim.getObjectHandle('/display7')
    }
    displayParameters = {
        {0, 0, 1, 0, 0, 1, 0},
        {1, 0, 1, 1, 1, 0, 1},
        {1, 0, 1, 1, 0, 1, 1},
        {0, 1, 1, 1, 0, 1, 0},
        {1, 1, 0, 1, 0, 1, 1},
        {1, 1, 0, 1, 1, 1, 1},
        {1, 0, 1, 0, 0, 1, 0},
        {1, 1, 1, 1, 1, 1, 1}
    }
    goalEffectEnabled = false
    lastGoalPlayer = nil
    flashingState = false
    flashingTimer = 0
    goalHandle = sim.getObjectHandle('/goal')
    flashingDuration = 0.5
    flashingTimer = 0
    isFlashing = false
end
 
sensor = sim.getObject('./sensor')
initialPosBall = sim.getObjectPosition(sensor, -1)
ball = sim.getObject('/ball')
 
function sysCall_actuation()
    result = sim.readProximitySensor(sensor)
    if (result > 0) then
        score = score + 1
        sim.setObjectPosition(ball, -1, {0,0,0.25})
         
        -- Rotate the wheel joint by 36 degrees
        local currentAngle = sim.getJointPosition(wheelJoint)
        local targetAngle = currentAngle + math.rad(-36)
        sim.setJointTargetPosition(wheelJoint, targetAngle)
         
        for i = 1, 8 do
            sim.setObjectPosition(robot[i], -1, initialPos[i])
            sim.setObjectOrientation(robot[i], -1, initialOri[i])
        end
         
        goalEffectEnabled = true
        lastGoalPlayer = result
        flashingState = false
        flashingTimer = 0
    end
     
    if goalEffectEnabled then
        if not isFlashing then
            flashingTimer = flashingTimer + sim.getSimulationTimeStep()
            if flashingTimer >= flashingDuration then
                isFlashing = true
                sim.setShapeColor(goalHandle, nil, sim.colorcomponent_ambient_diffuse, {1, 0.9, 0})
            end
        elseif isFlashing then
            flashingTimer = flashingTimer + sim.getSimulationTimeStep()
            if flashingTimer >= flashingDuration + 3 then
                goalEffectEnabled = false
                isFlashing = false
                sim.setShapeColor(goalHandle, nil, sim.colorcomponent_ambient_diffuse, {0, 0, 0})
            end
        end
    end
end
 
function sysCall_cleanup()
    sim.setShapeColor(goalHandle, nil, sim.colorcomponent_ambient_diffuse,{0, 0, 0})
end
\end{lstlisting}
檢測與球碰撞的球員並顯示其背號在七段顯示器中\\
\begin{lstlisting}[language=Python, frame=single, numbers=left, captionpos=b, basicstyle=\ttfamily\small, showstringspaces=false, breaklines=true, tabsize=4, xleftmargin=15pt]
local robotHandles = {}
local ballHandle = -1
local displayHandles = {}
local displayParameters = {
    {0, 0, 1, 0, 0, 1, 0},
    {1, 0, 1, 1, 1, 0, 1},
    {1, 0, 1, 1, 0, 1, 1},
    {0, 1, 1, 1, 0, 1, 0},
    {1, 1, 0, 1, 0, 1, 1},
    {1, 1, 0, 1, 1, 1, 1},
    {1, 0, 1, 0, 0, 1, 0},
    {1, 1, 1, 1, 1, 1, 1}
}
 
function sysCall_init()
    for i = 1, 8 do
        robotHandles[i] = sim.getObjectHandle('car' .. i)
    end
    ballHandle = sim.getObjectHandle('ball')
 
    for i = 1, 7 do
        displayHandles[i] = sim.getObjectHandle('/display' .. i)
    end
end
 
function sysCall_actuation()
    for i = 1, 8 do
        local collision = sim.checkCollision(robotHandles[i], ballHandle)
        if collision ~= 0 then
            setDisplayDigit(i, i <= 4)
            break
        end
    end
end
 
function setDisplayDigit(digit, isRed)
    for i = 1, 7 do
        if displayParameters[digit][i] == 1 then
            if isRed then
                sim.setShapeColor(displayHandles[i], nil, sim.colorcomponent_ambient_diffuse, {1, 0, 0})
            else
                sim.setShapeColor(displayHandles[i], nil, sim.colorcomponent_ambient_diffuse, {0, 1, 0})
            end
        else
            sim.setShapeColor(displayHandles[i], nil, sim.colorcomponent_ambient_diffuse, {0, 0, 0})
        end
    end
end
\end{lstlisting}